\documentclass{article}
\usepackage{tocloft}

\renewcommand{\cftsecleader}{\cftdotfill{\cftdotsep}}
\renewcommand{\contentsname}{Indice}

\title{Architettura Client-Server UDP per Trasferimento File}
\author{Marco Costantini \and Daniele Pancottini}
\date{27 Luglio 2022}

\begin{document}

\maketitle
\newpage

%%% TABLE OF CONTENTS  %%%
\tableofcontents

\newpage
\section{Analisi del Problema}
\subsection{Descrizione}

\newpage
\section{Design}
\subsection{Protocollo}
\subsection{Comandi}

\newpage
\section{Sviluppo}
\subsection{Client}
La classe Client definisce il socket UDP per la comunicazione con il server.
Tramite un menù di scelta vengono inviati i comandi al server (con il comando  \textit{sendto}) che li gestirà e restituirà il risultato al client.
Ricevuto il risultato dal server (tramite il comando \textit{recvfrom}) eseguirà a seconda del comando inviato:
\begin{itemize}
    \item \textbf{List} stamperà il nome dei files ricevuti dal server.
    \item \textbf{Put} userà la funzione rdt per la gestione dei pacchetti.
    \item \textbf{Get} controllerà l'effettiva esistenza del file e successivamente userà la funzione rdt per la gestione dei pacchetti.
\end{itemize}

\subsection{Server}
\subsection{RDT Handler}

\newpage
\section{Guida Utente}
\subsection{Client}
\subsection{Server}

%%% EOF TABLE OF CONTENTS %%%

\end{document}